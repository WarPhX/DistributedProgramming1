\documentclass[12pt]{book}
\usepackage{listings}
\lstset{numbers=left,numberstyle=\tiny,stepnumber=1,numbersep=5pt,numberblanklines=false,showstringspaces=false}
\lstset{frame=single} 
\newtheorem{exemplu}{Exemplul}[section]
\begin{document}

\chapter{Testare cu \textit{junit}}

Verificarea / testarea automat\u{a} a programelor Java se poate
face cu produsul informatic \textit{junit.}
Deseori se formuleaz\u{a} probleme de test ale c\u{a}ror
rezultate sunt cunoscute, cu rolul de a  verifica func\c{t}ionarea  
unui program de rezolvare, pentru depistarea unor gre\c{s}eli. 

S-a dezvoltat \c{s}i o metodologie de lucru \textit{Test Driven Development - (TDD)} care presupune pentru orice
clas\u{a} elaborat\u{a} realizarea unui program de testare, chiar a priori.

Un alt produs care are acela\c{s}i scop de verificare a rezultatelor este \textit{TestNG.}

\textit{junit} permite verificarea automat\u{a} a rezultatelor furnizate de
un program, pentru o mul\c{t}ime de date de test.

\textbf{Instalarea produsului} const\u{a} din dezarhivarea fi\c{s}ierului desc\u{a}rcat
\^{\i}ntr-un catalog \texttt{JUNIT\_HOME}.
Pentru compilare \c{s}i execu\c{t}ie, variabila de sistem \texttt{classpath}
trebuie s\u{a} con\c{t}in\u{a} referin\c{t}a \texttt{JUNIT\_HOME$\backslash$junit-*.jar.}

\textbf{Utilizarea produsului} \^{\i}ntr-un program Java const\u{a} din:
\begin{enumerate}
\item
Declararea resurselor pachetului \textit{junit} prin\\
\texttt{import org.junit.*;}\\
\texttt{import static org.junit.Assert.*;}
\item
Declararea clasei cu testele \textit{junit} - uzual \^{\i}n metoda \texttt{main.}\\
\texttt{org.junit.runner.JUnitCore.main("}\textit{AppClass}\texttt{");}
\item
Eventuale opera\c{t}ii necesare \^{\i}nainte sau dup\u{a} efectuarea testelor se precizeaz\u{a} respectiv,
\^{\i}n c\^{a}te o metod\u{a} care a fost declarat\u{a} cu adnotarea \texttt{@org.junit.Before} \c{s}i
respectiv, \texttt{@org.junit.After.}
\item
Testele se definesc \^{\i}n metode declarate cu adnotarea \texttt{@org.junit.Test.}

Clasa \texttt{Assert} posed\u{a} metodele de verificare ale unui rezultat:
\begin{itemize}
\item
\texttt{static void assertEquals(}Tip \textit{a\c{s}teptat}, Tip \textit{actual}\texttt{)}

unde Tip poate fi \texttt{double, int, long, Object.}
\item
\texttt{static void assertEquals(double} \textit{a\c{s}teptat,} \texttt{double} \textit{actual,} \texttt{double} \textit{delta}\texttt{)}

Testul reu\c{s}e\c{s}te dac\u{a} $|$\textit{a\c{s}teptat-actual}$|<delta.$
\item
\texttt{static void assertArrayEquals(}Tip[\ ] \textit{a\c{s}teptat}, Tip[\ ] \textit{actual}\texttt{)}

unde Tip poate fi \texttt{byte, char, int, long, short, Object.}
\item
\texttt{static void assertTrue(boolean} \textit{condi\c{t}ie}\texttt{)}
\item
\texttt{static void assertFalse(boolean} \textit{condi\c{t}ie}\texttt{)}
\item
\texttt{static void assertNull(Object} \textit{object}\texttt{)}
\item
\texttt{static void assertNotNull(Object} \textit{object}\texttt{)}
\end{itemize}
%Fiecare din metodele anterioare este dublat\u{a} de o metod\u{a} av\^{a}nd ca
%prim argument o variabil\u{a} de tip \texttt{String,} 
\end{enumerate}
\^{I}n cazul exemplului  
\scriptsize
\lstset{language=Java}
\begin{lstlisting}
import org.junit.*;
import static org.junit.Assert.*;

public class Exemplu{
  public double rezultat=1.0;
  public double eps=1e-6;
  
  double getValue(){
    return 1.0000001;
  }
  
  @Test 
  public void test(){   
    assertEquals(rezultat,getValue(),eps);
  }
  
  public static void main(String[] args){
    org.junit.runner.JUnitCore.main("Exemplu");
  }
}
\end{lstlisting}
\normalsize
se ob\c{t}ine
\scriptsize
\begin{verbatim}
JUnit version 4.5
.
Time: 0.03

OK (1 test)
\end{verbatim}
\normalsize

\begin{exemplu} Testarea clasei App ( \ref{appSocketCmmdc}).
\end{exemplu}

\scriptsize
\lstset{language=Java}
\begin{lstlisting}
package server;
import org.junit.*;
import static org.junit.Assert.*;

public class TestApp{
  private App app;
  
  @Before
  public void initializare(){
    app=new App(); 
  }
  
  @Test 
  public void test(){  
    assertEquals(8,app.cmmdc(56,24));
  }
  
  
  public static void main(String[] args){
    org.junit.runner.JUnitCore.main("server.TestApp");
  }
}
\end{lstlisting}
\normalsize

\begin{exemplu} Testarea clasei MyMServer ( \ref{appMyMServer}).
\end{exemplu}

\scriptsize
\lstset{language=Java}
\begin{lstlisting}
package server.impl;
import server.*;
import org.junit.*;
import static org.junit.Assert.*;
import java.net.ServerSocket;
import iserver.IMyMServer;

public class TestMyMServer{
  private IMyMServer obj;
  
  @Before
  public void initializare(){
    obj=new MyMServer(); 
  }
  
  @Test 
  public void test(){ 
    int port=7999;  
    Object result=obj.getServerSocket(port);
    assertNotNull("Must not return a null response",result);
    assertEquals(ServerSocket.class,result.getClass());
  }
  
  public static void main(String[] args){
    org.junit.runner.JUnitCore.main("server.impl.TestMyMServer");
  }
}
\end{lstlisting}
\normalsize
\end{document}